\chapter{Conclusioni}
\label{cap:conclusioni}

\section{Raggiungimento degli obiettivi}
Per quanto possibile sono stati completati tutti e tre gli obiettivi obbligatori.
Come si può notare dai grafici riportati alla fine dello scorso capitolo per i documenti HTML e Docx c'è stato un notevole miglioramento della qualità delle risposte da parte del RALM.
Per i Pdf invece il miglioramento è stato discreto.
Questa differenza è probabilmente dovuta al fatto che non riuscendo a convertire correttamente il file pdf in XHTML si perdono informazioni sulla struttura del contenuto.
Per esempio nei chunk non è possibile individuare i titoli superiori in ordine gererchico rispetto a un paragrafo.

\section{Conoscenze acquisite}
\begin{itemize}
    \item Python;
    \item Utilizzo di tool di estrazioni di testo/elementi da documenti (Tika, python-docx, pdfplumber, pandas);
    \item Conoscenza su NLP e LLM;
    \item Utilizzo di modelli per Chat-Completion;
    \item Utilizzo di tool per manipolazione di dati (pandas);
    \item Utilizzo di tool per manipolazione di codice XHTML (BeautifulSoup);
    \item Metodo di lavoro.
\end{itemize}

\section{Valutazione personale}
Per quanto mi riguarda sono molto soddisfatto del mio percorso di stage svolto.
Durante questo periodo ho avuto l'opportunità di approfondire le mie conoscienze nel campo del LLM (dei RALM in particolare).
Questa esperienza mi ha offerto una visione pratica del lavore nel settore a tutti gli effettie mi ha permesso di mettere in pratica ciò che ho imparato durante gli studi.

\\
Oltre ad ampliare le mie conoscenze tecniche, ho anche sviluppato la mie abilità nel problem solving e nella collaborazione.

In conclusione posso dire che mi ha permesso di capire che quello che voglio fare è continuare a studiare le intelligenze artificiali, è un campo che mi affascina molto.
Sono sicuro che quanto svolto mi sarà sicuramente utile in futuro.
