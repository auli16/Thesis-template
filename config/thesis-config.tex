% Variables
\newcommand{\myName}{Andrea Auletta}
\newcommand{\myTitle}{Sistema di estrazione di testo per RALM}
\newcommand{\myDegree}{Tesi di laurea}
\newcommand{\myUni}{Università degli Studi di Padova}
\newcommand{\myFaculty}{Corso di Laurea in Informatica}
\newcommand{\myDepartment}{Dipartimento di Matematica ``Tullio Levi-Civita''}
\newcommand{\profTitle}{Prof.}
\newcommand{\myProf}{Alessandro Sperduti}
\newcommand{\myLocation}{Padova}
\newcommand{\myAA}{2022-2023}
\newcommand{\myTime}{Luglio 2023}

% PDF/A filecontents
\RequirePackage{filecontents}
\begin{filecontents*}{\jobname.xmpdata}
  \Title{Document’s title}
  \Author{Author’s name}
  \Language{it-IT}
  \Subject{The abstract, or short description.}
  \Keywords{keyword1\sep keyword2\sep keyword3}
\end{filecontents*}

% Page format settings
% see: http://wwwcdf.pd.infn.it/AppuntiLinux/a2547.htm
\setlength{\parindent}{14pt}    % first row indentation
\setlength{\parskip}{0pt}       % paragraphs spacing

% Acronyms
\newacronym[description={\glslink{RALM}{Retrieval Augmented Language Model}}]
    {RALM}{RALM}{Retrieval Augmented Language Model}

\newacronym[description={\glslink{LLM}{Large Language Model}}]
    {LLM}{LLM}{Large Language Model}

\newacronym[description={\glslink{NLP}{Natural Language Processing}}]
    {NLP}{NLP}{Natural Language Processing}

\newacronym[description={\glslink{BM25}{Okapi Best Matching 25}}]
    {BM25}{BM25}{Okapi Best Matching 25}

\newacronym[description={\glslink{RRF}{Reciprocal Rank Fusion}}]
    {RRF}{RRF}{Reciprocal Rank Fusion}

\newacronym[description={\glslink{TQA}{Table Question Answering}}]
    {TQA}{TQA}{Table Question Answering}

% Glossary entries
\newglossaryentry{Bag-of-Words} {
    name={Bag-of-Words},
    description={Rappresentazione semplificata di un documento o di un testo in cui si ignora l'ordine delle parole e si considera la presenza o l'assenza dei vari termini}
}

\newglossaryentry{question answering}{
    name={Question answering},
    description={Campo dell'informatica e dell'intelligenza artificiale che si occupa di sviluppare sistemi in grado di comprendere e rispondere a domande poste dagli utenti in linguaggio naturale}
}

\newglossaryentry{Parsing}{
    name = {Parsing},
    description={In informatica, il parsing è la tecnica che permette di estrapolare, decomporre e comprendere la struttura sintattica e semantica delle informazioni significative}
}

\newglossaryentry{chunk}{
    name = {Chunk},
    description={Piccola porzione di testo estratta da un documento}
}

\newglossaryentry{sliding window}{
    name = {Sliding window},
    description = {Finestra mobile di dimensioni fisse che scorre attraverso porzioni di testo (o dati), consentendo l'analisi di sotto-sequenze sovrapposte per estrarre informazioni}
}

\newglossaryentry{Chat-Completion Model}{
    name = {Chat-Completion Model},
    description = {Modello di generazione del linguaggio artificiale che viene utilizzato per completare o generare testo in linguaggio naturale all'interno di una conversazione}
}

\newglossaryentry{token}{
    name = {Token},
    description = {Singola unità linguistica o elemento individuale all'interno di un testo che può rappresentare per una parola, un simbolo di punteggiatura o anche una parte di una parola}
}
\makeglossaries

\bibliography{appendix/bibliography}

\defbibheading{bibliography} {
    \cleardoublepage
    \phantomsection
    \addcontentsline{toc}{chapter}{\bibname}
    \chapter*{\bibname\markboth{\bibname}{\bibname}}
}

% Spacing between entries
\setlength\bibitemsep{1.5\itemsep}

\DeclareBibliographyCategory{opere}
\DeclareBibliographyCategory{web}

\addtocategory{opere}{womak:lean-thinking}
\addtocategory{web}{site:agile-manifesto}

\defbibheading{opere}{\section*{Riferimenti bibliografici}}
\defbibheading{web}{\section*{Siti Web consultati}}


\captionsetup{
    tableposition=top,
    figureposition=bottom,
    font=small,
    format=hang,
    labelfont=bf
}

% Images path
\graphicspath{{images/}}

\hypersetup{
    %hyperfootnotes=false,
    %pdfpagelabels,
    colorlinks=true,
    linktocpage=true,
    pdfstartpage=1,
    pdfstartview=,
    breaklinks=true,
    pdfpagemode=UseNone,
    pageanchor=true,
    pdfpagemode=UseOutlines,
    plainpages=false,
    bookmarksnumbered,
    bookmarksopen=true,
    bookmarksopenlevel=1,
    hypertexnames=true,
    pdfhighlight=/O,
    %nesting=true,
    %frenchlinks,
    urlcolor=webbrown,
    linkcolor=RoyalBlue,
    citecolor=webgreen
    %pagecolor=RoyalBlue,
}

% Delete all links and show them in black
\if \isprintable 1
    \hypersetup{draft}
\fi

% Itemize symbols
%\renewcommand{\labelitemi}{$\ast$}
%\renewcommand{\labelitemi}{$\bullet$}
%\renewcommand{\labelitemii}{$\cdot$}
%\renewcommand{\labelitemiii}{$\diamond$}
%\renewcommand{\labelitemiv}{$\ast$}

% Listings setup
\lstset{
    language=[LaTeX]Tex,%C++,
    keywordstyle=\color{RoyalBlue}, %\bfseries,
    basicstyle=\small\ttfamily,
    %identifierstyle=\color{NavyBlue},
    commentstyle=\color{Green}\ttfamily,
    stringstyle=\rmfamily,
    numbers=none, %left,%
    numberstyle=\scriptsize, %\tiny
    stepnumber=5,
    numbersep=8pt,
    showstringspaces=false,
    breaklines=true,
    frameround=ftff,
    frame=single
}

\definecolor{webgreen}{rgb}{0,.5,0}
\definecolor{webbrown}{rgb}{.6,0,0}

% \omiss produces '[...]'
\newcommand{\omissis}{[\dots\negthinspace]}

% Hyphenation rules
\hyphenation {
    ma-cro-istru-zio-ne
    gi-ral-din
}

\newcommand{\sectionname}{sezione}
\addto\captionsitalian{\renewcommand{\figurename}{Figura}
                       \renewcommand{\tablename}{Tabella}}

\newcommand{\glsfirstoccur}{\ap{{[g]}}}

\newcommand{\intro}[1]{\emph{\textsf{#1}}}

% Risks environment
\newcounter{riskcounter}                % define a counter
\setcounter{riskcounter}{0}             % set the counter to some initial value

%%%% Parameters
% #1: Title
\newenvironment{risk}[1]{
    \refstepcounter{riskcounter}        % increment counter
    \par \noindent                      % start new paragraph
    \textbf{\arabic{riskcounter}. #1}   % display the title before the content of the environment is displayed
}{
    \par\medskip
}

\newcommand{\riskname}{Rischio}

\newcommand{\riskdescription}[1]{\textbf{\\Descrizione:} #1.}

\newcommand{\risksolution}[1]{\textbf{\\Soluzione:} #1.}

% Use case environment
\newcounter{usecasecounter}             % define a counter
\setcounter{usecasecounter}{0}          % set the counter to some initial value

%%%% Parameters
% #1: ID
% #2: Nome
\newenvironment{usecase}[2]{
    \renewcommand{\theusecasecounter}{\usecasename #1}  % this is where the display of
                                                        % the counter is overwritten/modified
    \refstepcounter{usecasecounter}             % increment counter
    \vspace{10pt}
    \par \noindent                              % start new paragraph
    {\large \textbf{\usecasename #1: #2}}       % display the title before the
                                                % content of the environment is displayed
    \medskip
}{
    \medskip
}

\newcommand{\usecasename}{UC}

\newcommand{\usecaseactors}[1]{\textbf{\\Attori Principali:} #1. \vspace{4pt}}
\newcommand{\usecasepre}[1]{\textbf{\\Precondizioni:} #1. \vspace{4pt}}
\newcommand{\usecasedesc}[1]{\textbf{\\Descrizione:} #1. \vspace{4pt}}
\newcommand{\usecasepost}[1]{\textbf{\\Postcondizioni:} #1. \vspace{4pt}}
\newcommand{\usecasealt}[1]{\textbf{\\Scenario Alternativo:} #1. \vspace{4pt}}

% Namespace description environment
\newenvironment{namespacedesc}{
    \vspace{10pt}
    \par \noindent  % start new paragraph
    \begin{description}
}{
    \end{description}
    \medskip
}

\newcommand{\classdesc}[2]{\item[\textbf{#1:}] #2}
