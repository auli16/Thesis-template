\chapter{Fase di studio e di ricerca preliminare}
\label{cap:analisi preliminare}

\intro{In questo capitolo verranno illustrate le varie metodologie individuate per poter rendere di maggiore qualità le risposte fornite dal RALM e le diverse tecnologie utlizzate}\\

\section{Individuazione del problema}
I documenti, prima di essere passati al RALM, devono essere rielaborati in modo tale da renderli più "comprensibili".
Il contenuto viene quindi diviso in porzioni di testo più piccole chiamate "chunk". A ogni domanda vengono analizzati tutti i 
diversi chunk e ad ognuno di essi viene assegnato uno "score" che stabilirà quale di loro ha il contenuto più adatto da cui 
poter ricavare la risposta.
I problemi principali affrontati in questo stage sono stati i seguenti:
\begin{itemize}
    \item Linearizzazione delle tabelle: gli strumenti utilizzati per estrapolare il testo non permettevano di linearizzare 
    la tabella in una maniera tale da renderla "leggibile" al RALM;
    \item Individuazione della struttura del contenuto e suddivisione sensata all'interno dei chunk: 
    inizialmente il testo veniva spezzato in punti poco sensati raggiunta una determinata quantità di caratteri già impostata e inserito all'interno di diversi chunk.
    C'era quindi la possibilità di separare delle informazioni che sarebbe stato meglio avere all'interno dello stesso chunk (come ad esempio un paragrafo e il suo titolo).
\end{itemize}

\subsection{Linearizzazione delle tabelle}
Una tabella può assumere moltissime strutture e per questo abbiamo deciso di considerare solamente le tabelle che avessero come prima riga 
un'intesazione orizzontale e nelle righe successive i vari dati.

\begin{table}[H]
    \centering
    \begin{tabular}{|p{3cm} |p{2cm} |p{2cm}| p{2cm}| p{2cm}|}
        \hline
        Cibo & Quantità & Energia(KCal) & Carboidrati(g) & Proteine(g) \\
        \hline
        Pennette rigate & 100g & 359 & 71 & 13 \\
        \hline
        Latte & 100ml & 47 & 4,9 & 3,2 \\
        \hline
        Banana & 100g & 89 & 23 & 1,1 \\
        \hline
    \end{tabular}
    \caption{Esempio di tabella presa in considerazione (valori approssimativi)}
    \label{tab:esempio-cibo}
\end{table}

Come specificato precedentemente le tabelle al momento dell'estrazione venivano linearizzate 
perdendo alcune informazioni necessarie per la lettura effettuata dal RALM. 

Per esempio la tabella \ref{tab:esempio-cibo} verrebbe linearizzata in questa maniera:
\begin{tcolorbox}[colback=white, colframe=black]
    Cibo Quantità Energia(KCal) Carboidrati(g) Proteine(g) Pennette rigate  100g  359  71  13 Latte 100ml 47 4,9 3,2 Banana  100g 89 23 1,1
\end{tcolorbox}

La tabella linearizzata perde quindi le informazioni sulla struttura e come risultato abbiamo una serie di valori posti senza avere troppo senso in fase di lettura per un RALM.

Dopo un attenta ricerca effettuata su vari documenti scientifici sono riuscito ad individuare un modo semplice ed efficace per mantenere 
l'informazione nella tabella linearizzata e la sua struttura:
\begin{itemize}
    \item All'inizio di ogni riga viene scritto "Riga n->" dove n sta per il numero della riga;
    \item Per ogni cella presente nella tabella vengono concatenati il valore dell'intestazione della colonna dov'è presente il valore e il valore della cella separati dal carattere ":";
    \item Ogni cella viene poi separata dall'altra con il carattere "|".
\end{itemize} 

Quindi la tabella \ref{tab:esempio-cibo} viene linearizzata in questo modo:
\begin{tcolorbox}[colback=white, colframe=black]
    Riga0->Cibo: Pennette rigate|Quantità:100g|Energia(KCal):359|Carboidrati(g): \\
    71|Proteine(g):13| Riga1->Cibo: Latte|Quantità:100ml|Energia(KCal):47| \\
    Carboidrati(g):4,9|Proteine(g):3,2| Riga2->Cibo: Banana|Quantità:100g \\
    |Energia(KCal):89|Carboidrati(g):23|Proteine(g):1,1|
\end{tcolorbox}




\section{Analisi preventiva dei rischi}

%Durante la fase di analisi iniziale sono stati individuati alcuni possibili rischi a cui si potrà andare incontro.
%Si è quindi proceduto a elaborare delle possibili soluzioni per far fronte a tali rischi.\\

\begin{risk}{Performance del simulatore hardware}
    \riskdescription{le performance del simulatore hardware e la comunicazione con questo potrebbero risultare lenti o non abbastanza buoni da causare il fallimento dei test}
    \risksolution{coinvolgimento del responsabile a capo del progetto relativo il simulatore hardware}
    \label{risk:hardware-simulator} 
\end{risk}