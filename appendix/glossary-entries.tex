% Acronyms
\newacronym[description={\glslink{RALM}{Retrieval Augmented Language Model}}]
    {RALM}{RALM}{Retrieval Augmented Language Model}

\newacronym[description={\glslink{LLM}{Large Language Model}}]
    {LLM}{LLM}{Large Language Model}

\newacronym[description={\glslink{NLP}{Natural Language Processing}}]
    {NLP}{NLP}{Natural Language Processing}

\newacronym[description={\glslink{BM25}{Okapi Best Matching 25}}]
    {BM25}{BM25}{Okapi Best Matching 25}

\newacronym[description={\glslink{RRF}{Reciprocal Rank Fusion}}]
    {RRF}{RRF}{Reciprocal Rank Fusion}

\newacronym[description={\glslink{TQA}{Table Question Answering}}]
    {TQA}{TQA}{Table Question Answering}

% Glossary entries
\newglossaryentry{Bag-of-Words} {
    name={Bag-of-Words},
    description={Rappresentazione semplificata di un documento o di un testo in cui si ignora l'ordine delle parole e si considera la presenza o l'assenza dei vari termini}
}

\newglossaryentry{question answering}{
    name={Question answering},
    description={Campo dell'informatica e dell'intelligenza artificiale che si occupa di sviluppare sistemi in grado di comprendere e rispondere a domande poste dagli utenti in linguaggio naturale}
}

\newglossaryentry{Parsing}{
    name = {Parsing},
    description={In informatica, il parsing è la tecnica che permette di estrapolare, decomporre e comprendere la struttura sintattica e semantica delle informazioni significative}
}

\newglossaryentry{chunk}{
    name = {Chunk},
    description={Piccola porzione di testo estratta da un documento}
}

\newglossaryentry{sliding window}{
    name = {Sliding window},
    description = {Finestra mobile di dimensioni fisse che scorre attraverso porzioni di testo (o dati), consentendo l'analisi di sotto-sequenze sovrapposte per estrarre informazioni}
}

\newglossaryentry{Chat-Completion Model}{
    name = {Chat-Completion Model},
    description = {Modello di generazione del linguaggio artificiale che viene utilizzato per completare o generare testo in linguaggio naturale all'interno di una conversazione}
}

\newglossaryentry{token}{
    name = {Token},
    description = {Singola unità linguistica o elemento individuale all'interno di un testo che può rappresentare per una parola, un simbolo di punteggiatura o anche una parte di una parola}
}

\newglossaryentry{overlap}{
    name = {Overlap},
    description = {Sovrapposizione di elementi. Nel caso del chunking, l'overlap corrisponde nella parte finale e nella parte iniziale di due chunk consecutivi}
}

\newglossaryentry{strategy}{
    name = {Strategy},
    description = {Pattern che tenta di isolare un algoritmo all'interno di un oggetto, in maniera tale da risultare utile in quelle situazioni dove sia necessaria modificare dinamicamente l'algoritmo stesso}
}

\newglossaryentry{design pattern}{
    name = {Design pattern},
    description = {Soluzione progettuale generale ad un problema riccorrente, serve per risolvere problemi di progettazione che possono presentarsi diverse volte}
}