\chapter{Conclusioni}
\label{cap:conclusioni}

\section{Raggiungimento degli obbiettivi}
Per quanto possibile sono stati completati tutti e tre gli obiettivi obbligatori.
Come si può notare dai grafici riportati alla fine dello scorso capitolo per i documenti HTML e Docx c'è stato un notevole miglioramento della qualità delle risposte da parte del RALM.
Per i Pdf invece il miglioramento è stato discreto.
Questa differenza è dovuta al fatto che non riuscendo a convertire correttamente il file pdf in XHTML si perdono informazioni sulla struttura del contenuto.
Per esempio nei chunk non è possibile individuare i titoli effettivi di un paragrafo.

\section{Conoscenze acquisite}
Come conoscenze teoriche ho approfondito meglio concetti su NLP e LLM come il chunking, questo grazie allo studio del RALM.
Di interessante, ho appreso anche il funzionamento del ranking e, quindi, come vengono assegnati gli score ai chunk tramite l'algoritmo di ricerca ibrida.
Al livello pratico, invece, durante questo percorso di stage ho sicuramente accresciuto le mie conoscenze sul linguaggio di programmazione Python che prima conoscevo in maniera abbastanza basilare, ora invece conosco anche diversi strumenti utili per l'estrazione di informazoni da documenti come tika, pdfplumber, Pandas e python-docx e strumenti per manipolare codice XHTML come BeautifulSoup.
Ho imparato anche come utilizzare i modelli che fornisce OpenAI e come utlizzare il motore di ricerca Weaviate per fornire gli score ai chunk.
Al livello personale invece ho appreso come migliorare il mio metodo di lavoro sotto il punto di vista dell'organizzazione, del problem solving e della collaborazione.

\section{Materiale prodotto}
\subsection{Documentazione}

Qui di seguito viene riportata la tabella che tratta brevemente della documentazione prodotta durante lo stage.
\begin{table}[H]
    \centering
    \begin{tabular}{|p{3cm} |p{8cm} |}
        \hline
        \textbf{Titolo} & \textbf{Descrizione} \\
        \hline
        Appunti su tabelle & Contiene esempi su linearizzazioni di tabelle utili per agevolare il TQA. \\
        \hline
        Estrazione delle tabelle da documenti & Contiene prove ed esempi che riguardano il funzionamento dei vari tool per l'estrazioni delle tabelle (Pandas, pdfplumber, python-docx). \\
        \hline
        Sostituzione tabella del documento con tabella linearizzata & Contiene informazioni su come il contenuto dei documenti viene convertito da Tika e le varie soluzioni di replace applicate nei casi in cui converte correttamente e non il contenuto. \\
        \hline
        Progettazione & Contiene tutte le informazioni riguardanti la progettazione software del prodotto e le motivazioni sulle scelte effettuate. \\
        \hline
        Idee chunking & Contiene diverse idee su come migliorare il chunking e le motivazioni a favore della scelta applicata. \\
        \hline
        Test & Contiene i risultati raccolti dopo aver posto le domande sulle cinque pagine di Wikipedia prese in considerazione per i tre formati. \\
        \hline
    \end{tabular}
    \caption{Tabella dei documenti prodotti durante lo stage.}
\end{table}

\subsection{Codice sviluppato}
Qui di seguito viene riportata la tabella che descrive brevemente i file di codice sviluppati durante il progetto.
\begin{table}[H]
    \centering
    \begin{tabular}{|p{4cm} |p{1cm} | p{2cm} |p{6cm}|}
        \hline
        \textbf{Titolo file} & \textbf{Righe} & \textbf{Formato} & \textbf{Descrizione}\\
        \hline
        AbstractExtractionTable & 65 & Python & Classe astratta che si occuppa della parte di algoritmo che gestisce l'etrazione, della linearizzazione e della sostituzione delle tabelle all'interno dei documenti. \\
        \hline
        ExtractionTableHTML & 30 & Python & Implementazione della classe astratta AbstractExtractionTable, si occupa delle operazioni sulle tabelle per i file HTML. \\
        \hline
        ExtractionTableDocx & 24 & Python & Implementazione della classe astratta AbstractExtractionTable, si occupa delle operazioni sulle tabelle per i file Docx. \\
        \hline
        ExtractionTablePdf & 76 & Python & Implementazione della classe astratta AbstractExtractionTable, si occupa delle operazioni sulle tabelle per i file Pdf. \\
        \hline
        ExtractionTableDefault & 22 & Python & Implementazione della classe astratta AbstractExtractionTable, si occupa delle operazioni sulle tabelle per i file che non hanno una classe specifica che implementa AbstractExtractionTable. \\
        \hline
        ExtractText & 62 & Python & Contiene le funzioni utili alla preparazione del contenuto per il chunking. \\
        \hline
        FactoryExtractText & 22 & Python & Contiene la funzione "builder" che istanzia un oggetto di tipo ExtractionTable del formato del documento sul quale si sta lavorando. \\
        \hline 
        Section & 105 & Python & Contiene le funzioni per convertire codice XHTML in un albero "Sezione". \\
        \hline
        Chunking & 303 & Python & Contiene le funzioni per convertire documento in una lista di chunk, contiene funzioni per lavorare con i chunking. \\
        \hline
        ChunkExample &  & Jupyter Notebook & Contiene esempi sul funzionamento del codice sviluppato, partendo dalla sostituzione delle tabelle al chunking. \\
        \hline 
        ExtractPdfTest & & Jupyter Notebook & Contiene esempi sul funzionamento di alcuni tool di estrazione di tabelle dai documenti Pdf. \\
        \hline
        VectorStoreGeneration & & Jupyter Notebook & Prototipo aziendale RALM moodificato con le aggiunte riguardanti le tabelle e il chunking. \\
        \hline 
    \end{tabular}
    \caption{Tabella che riguarda i file di codice prodotti durante lo stage.}
\end{table}

\section{Consuntivo}
Qui di seguito viene riportata la tabella che presenta il consuntivo: 
\begin{table}[H]
    \centering
    \begin{tabular}{p{2cm} p{8cm} p{2cm}}
        \hline
        Numero attività & Attività & Ore effettuate \\
        \hline
        1 & Studio introduttivo su Natural Language Processing e Large Language Model & 16 \\
        \hline
        2 & Studio delle tecniche di estrazione di testo e dei principali tool nell'ambito dell'NLP & 16 \\
        \hline
        3 & Studio dell'attuale implementazione del chatbot basato su retrueval-augmented LLM & 16 \\
        \hline
        4 & Analisi dei requisiti con studio delle casistiche da gestire & 30 \\
        \hline
        5 & Progettazione delle varie componenti richieste nel paragrafo  & 66 \\
        \hline
        6 & Implementazione del software & 96 \\
        \hline
        7 & Test e sperimentazione del software & 24 \\
        \hline
        8 & Documentazione & 40 \\
        \hline
    \end{tabular}
    \caption{Tabella consuntivo.}
\end{table}

Rispetto al preventivo presentato nella tabella \ref{tab:preventivo} sono state effettuate alcune ore in meno dovute alla visualizzazione di alcuni corsi online che riguradavano norme aziendali, comunque le quantità orarie non discostano in maniera significativa. 

Gli obbiettivi obbligatori presentati nella sezione \ref{subsec:requisiti} sono stati tutti raggiunti, mentre, l'obbiettivo desiderabile che riguradava l'interpretazione delle immagini non è stato completato per mancanza di tempo.

\section{Valutazione personale}
Per quanto mi riguarda sono molto soddisfatto del percorso di stage svolto.
Durante questo periodo ho avuto l'opportunità di approfondire le mie conoscienze nel campo del LLM (dei RALM in particolare).
Questa esperienza mi ha offerto una visione pratica del lavoro nel settore a tutti gli effetti e mi ha permesso di mettere in pratica ciò che ho imparato durante gli studi.


Oltre ad ampliare le mie conoscenze tecniche, ho anche sviluppato la mie abilità nel problem solving e nella collaborazione.

In conclusione posso dire che mi ha permesso di capire che quello che voglio fare è continuare a studiare le intelligenze artificiali, è un campo che mi affascina molto.
Sono sicuro che quanto svolto mi sarà sicuramente utile in futuro.
