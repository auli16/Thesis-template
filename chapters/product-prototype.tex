\chapter{Progettazione e codifica}
\label{cap:progettazione-codifica}

\intro{Breve introduzione al capitolo}\\

\section{Tecnologie e strumenti}
\label{sec:tecnologie-strumenti}

Di seguito viene data una panoramica delle tecnologie e strumenti utilizzati.
\subsection{Python}
Python offre diverse librerie utili per l'apprendimento automatico, il trattamento del linguaggio naturale e molto altro ancora.
\subsection{Tika}
\label{subsec:tika}
Tika è un tool che permette di estrarre dai file i suoi metadata (come titolo del file, autore e altro) e il suo contenuto sottoforma di testo (strutturato e non).
Riesce ad estrarre i dati da molti formati tra cui Pdf, docx e html.
\subsection{Pandas}
Pandas è una libreria che fornisce strutture dati e strumenti per la manipolazione e l'analisi dei dati.
Il primo passo importante dell'estrazione delle tabelle dai vari documenti è stato quello di avere un DataFrame a disposizione con i valori di quest'ultime. 
Pandas è stato utilizzato per l'estrazione delle tabelle dai file in formato html e docx.
\subsection{Camelot}
Camelot è una libreria per Python che fornisce funzionalità per l'estrazione automatizzata di tabelle dai documenti PDF.
Camelot, una volta individuata la tabella in un documento, estrae i dati e li restituisce come un DataFrame di Pandas.
\subsection{Python-docx}
Python-docx è una libreria che consente di creare, modificare e leggere documenti Microsoft Word.
Grazie a questa libreria è possibile manipolare le tabelle presenti all'interno dei file docx.
\subsection{BeautifulSoup}
BeautifulSoup è una libreria Python che facilita l'estrazione di dati da file HTML e XML.
\subsection{Weaviate}
Weaviate è un database vettoriale, un motore di ricerca basato sulla "vector similarity search".
Weaviate è utile quindi per la ricerca dei dati basata sulla loro semantica e sulle loro relazioni. 

\section{Ciclo di vita del software}
\label{sec:ciclo-vita-software}

\section{Progettazione}
\label{sec:progettazione}

\subsubsection{Namespace 1} %**************************
Descrizione namespace 1.

\begin{namespacedesc}
    \classdesc{Classe 1}{Descrizione classe 1}
    \classdesc{Classe 2}{Descrizione classe 2}
\end{namespacedesc}


\section{Design Pattern utilizzati}

\section{Codifica}
