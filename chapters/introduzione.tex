\chapter{Introduzione}
\label{cap:introduzione}

In questo capitolo viene spiegato com'è strutturato questo documento e vengono introdotti i RALM.
Inoltre viene presentata l'azienda presso quale è stato svolto lo stage e vengono 
fornite le motivazioni per i quale è nato progetto e il suo scopo

\section{Organizzazione del testo}

\begin{description}
    \item[{\hyperref[cap:processi-metodologie]{Il secondo capitolo}}] descrive ...
    
    \item[{\hyperref[cap:descrizione-stage]{Il terzo capitolo}}] approfondisce ...
    
    \item[{\hyperref[cap:analisi-requisiti]{Il quarto capitolo}}] approfondisce ...
    
    \item[{\hyperref[cap:progettazione-codifica]{Il quinto capitolo}}] approfondisce ...
    
    \item[{\hyperref[cap:verifica-validazione]{Il sesto capitolo}}] approfondisce ...
    
    \item[{\hyperref[cap:conclusioni]{Nel settimo capitolo}}] descrive ...
\end{description}

Riguardo la stesura del testo, relativamente al documento sono state adottate le seguenti convenzioni tipografiche:
\begin{itemize}
	\item gli acronimi, le abbreviazioni e i termini ambigui o di uso non comune menzionati vengono definiti nel glossario, situato alla fine del presente documento;
	\item per la prima occorrenza dei termini riportati nel glossario viene utilizzata la seguente nomenclatura: \emph{parola}\glsfirstoccur;
	\item i termini in lingua straniera o facenti parti del gergo tecnico sono evidenziati con il carattere \emph{corsivo}.
\end{itemize}

\section{RALM}
I Large Language Model(LLM) sono dei modelli di apprendimento automatico in grado di generare testi coerenti e informativi.
Hanno rivoluzionato il campo dei Natural Language Processing (NLP) vengono impiegati in diverse applicazioni, un esempio possono essere i chatbot.
Questi ultimi sfruttano la capacità dei LLM di interagire tramite il linguaggio naturale con gli utenti.
I Retrieval Augmented Language Model (RALM) non sono altro che dei LLM che permettono di utilizzare una fonte estera di conoscenza (come ad esempio una collezione di documenti)
per fornire informazioni aggiuntive al modello durante la generazione di testo.
Il modello effettua una query alla fonte esterna usando l'input dell'utente come chiave di ricerca e riceve una lista di documenti rilevanti.
Seleziona, poi, uno o pù documenti da usare come contesto aggiuntivo e li combina con l'input originale per produrre il testo finale.
Questo approccio permette di generate testi più informativi, accurati e diversificati sfruttando la conoscenza dovuta alla presenza della fonte esterna.

\section{Siav S.p.A.}

\begin{figure}[!h] 
    \centering 
    \includegraphics[width=0.5\columnwidth]{images/logoSiav.jpg} 
    \caption{Logo dell'azienda Siav S.p.A.}
\end{figure}
Siav S.p.A. è un’azienda informatica specializzata nella dematerializzazione, nella gestione elettronica dei documenti e nei processi digitali.
Fondata nel 1990 a Rubano (Padova) è oggi la prima azienda italiana nel settore dell’Enterprise Content Management, e offre software, soluzioni in cloud e servizi di outsourcing per la Gestione Elettronica dei Documenti, il Protocollo Informatico, il Workflow Management, la Fatturazione Elettronica e la Conservazione Digitale.

\section{Il progetto}

Per poter garantire la qualità delle risposte generate dal RALM è necessario che i documenti a disposizione siano in formato testuale e che il loro contenuto abbia tutta l'informazione 
neccessaria.
I documenti sono disponibili in diversi formati e spesso non sono costituiti semplicemente di testo non strutturato, ma presentano frequentemente vari
"elementi semantici" come tabelle, immagini e titoli.
Presentano quindi informazione che non era immediatamente estraibile e convertibile in testo utile ai fini dell'interazione col RALM.

\subsection{Obbiettivi}
Una prima versione di backend dei servizi di estrazione di testo era già disponibile e lo scopo di questo progetto era quello di migliorare proprio quest'ultimo raggiungendo in seguenti obbiettivi:

\begin{itemize}
    \item Obbligatori:
    \subitem Parsing ad-hoc per documenti in cui le componenti grafiche contribuiscono alla semantica (es. tabelle e immagini);
    \subitem Estrapolazione, per alcuni formati dove sia possibile, della struttura logica del documento (es. individuando titoli e paragrafi);
    \subitem Pulizia del testo estratto dai documenti (es. eliminazione delle componenti inutili come gli indici e i sommari).
    \item Desiderabili:
    \subitem Interpretazione delle immagini allo scopo di arricchire i chunk in cui sono contestualizzate. 
\end{itemize}

\subsection{Pianificazione delle attività}
\begin{table}[H]
    \centering
    \begin{tabular}{p{2cm} p{8cm} p{2cm}}
        \hline
        Numero attività & Attività & Ore previste \\
        \hline
        1 & Studio introduttivo su Natural Language Processing e Large Language Model & 16 \\
        \hline
        2 & Studio delle tecniche di estrazione di testo e dei principali tool nell'ambito dell'NLP & 16 \\
        \hline
        3 & Studio dell'attuale implementazione del chatbot basato su retrueval-augmented LLM & 16 \\
        \hline
        4 & Analisi dei requisiti con studio delle casistiche da gestire & 30 \\
        \hline
        5 & Progettazione delle varie componenti richieste nel paragrafo  & 70 \\
        \hline
        6 & Implementazione del software & 100 \\
        \hline
        7 & Test e sperimentazione del software & 24 \\
        \hline
        8 & Documentazione & 48 \\
        \hline
    \end{tabular}
    \caption{Tabella di pianificazione delle attività}
\end{table}

\subsection{Diagramma di Gantt delle attività}
Viene mostrato il diagramma di Gantt delle attività svolte durante le nove settimane dello stage.

\begin{figure}[H]
    \centering
    \begin{ganttchart}[
        expand chart=\textwidth,
        hgrid=true,
        vgrid=true
        ]{1}{9}
        \gantttitlelist{1,...,9}{1} \\
        \ganttbar{Studio NLP e LLM}{1}{1} \\
        \ganttbar{Studio tecniche e tool esistenti}{1}{2} \\
        \ganttbar{Studio attuale implementazione chatbot}{2}{2} \\
        \ganttbar{Analisi dei requisiti}{2}{3} \\
        \ganttbar{Progettazione}{3}{5} \\
        \ganttbar{Implementazione del software}{5}{8} \\
        \ganttbar{Test e sperimentazione del software}{8}{9} \\
        \ganttbar{Documentazione}{1}{9} 
    \end{ganttchart}
    \caption{Diagramma di Gantt delle attività}
\end{figure}
